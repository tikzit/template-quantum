\documentclass{article}
\usepackage[margin=3cm]{geometry}
\usepackage{tikzit}
\input{quantum.tikzstyles}
\input{quantum.tikzdefs}

\title{TikZiT Quantum Template}

\begin{document}

\maketitle

The standard way to use files produced by TikZiT is to place the \texttt{.tikz} files in the \texttt{figures} subdirectory, and include them via the \texttt{\textbackslash{}tikzfig} macro provided by \texttt{tikzit.sty}. This is essentially a wrapper for \texttt{\textbackslash{}input}. This macro expects the filename without an extension or \texttt{figures/}, so e.g. \texttt{\textbackslash{}tikzfig\{circ\}} will input the file \texttt{figures/circ.tikz}. You can also use \texttt{\textbackslash{}ctikzfig} as a shorthand for \texttt{\textbackslash{}tikzfig} wrapped in the \texttt{center} environment.

This quantum template contains TikZ styles to make quantum circuits:
\ctikzfig{circ}
ZX-diagrams:
\ctikzfig{zx}
and other handy stuff, like the Bloch sphere:
\ctikzfig{bloch}

\end{document}
