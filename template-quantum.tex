\documentclass{article}
\usepackage[margin=2cm]{geometry}
\usepackage{hyperref}

% "draft" enables (manual) caching of figures for faster builds
\usepackage[draft]{tikzit}
\input{quantum.tikzstyles}
\input{quantum.tikzdefs}

\title{TikZiT Quantum Template}
\author{}

\begin{document}

\maketitle

The standard way to use files produced by TikZiT is to place the \texttt{.tikz} files in the \texttt{figures} subdirectory, and include them via the \texttt{\textbackslash{}tikzfig} macro provided by \texttt{tikzit.sty}. This is essentially a wrapper for \texttt{\textbackslash{}input}. This macro expects the filename without an extension or \texttt{figures/}, so e.g. \texttt{\textbackslash{}tikzfig\{circ\}} will input the file \texttt{figures/circ.tikz}. You can also use \texttt{\textbackslash{}ctikzfig} as a shorthand for \texttt{\textbackslash{}tikzfig} wrapped in the \texttt{center} environment. For everything else you ever wanted to know about TikZiT and \texttt{tikzit.sty}, check out:

\begin{center}
  \url{https://tikzit.github.io}.
\end{center}

\section*{Loading styles in TikZiT}

Before you open one of the example figures in TikZiT, make sure you load the \texttt{.tikzstyles} file included in this directory. You can do that by clicking the icon that looks like a folder near the top-right of the main TikZiT window. This will give you a handful of styles to start with, and ensure that when you preview figures in TikZiT, they look the same as they do in this paper.

\section*{Using pre-built figures}

By passing the \texttt{[draft]} option to \texttt{tikzit.sty}, figures can now be pre-built using the included \texttt{Rakefile}, or any other way you like. This can substantially reduce build time if you have lots of figures. With the \texttt{draft} option active, \texttt{\textbackslash{}tikzfig\{FOO\}} will first search for a file called \texttt{cache/FOO.pdf} and include that before trying to build the tikz figure. To use the included build script, you'll need \texttt{rake} (\url{https://github.com/ruby/rake}). Then, you can pre-build simply by running \texttt{rake} in from the command line in the same directory as this paper. You can also listen for changes and rebuild figures on-demand with \texttt{rake listen}. If you change the name of the main file or add more tex files to your project, edit the appropriate options at the top of the included \texttt{Rakefile}.

\section*{The quantum template}

This quantum template contains TikZ styles to make quantum circuits, ZX-calculus diagrams, and other handy stuff like the Bloch sphere:
\[
\tikzfig{circ} \quad
\tikzfig{zx}
\tikzfig{bloch}
\]

\end{document}
